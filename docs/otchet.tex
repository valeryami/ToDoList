%%% Для сборки выполнить 2 раза команду: pdflatex <имя файла>

\documentclass[a4paper,12pt]{article}

\usepackage{ucs}
\usepackage[utf8x]{inputenc}
\usepackage[russian]{babel}
%\usepackage{cmlgc}
\usepackage{graphicx}
\usepackage{listings}
\usepackage{xcolor}
\usepackage{titlesec}
%\usepackage{courier}


\makeatletter
\renewcommand\@biblabel[1]{#1.}
\makeatother

\newcommand{\myrule}[1]{\rule{#1}{0.4pt}}
\newcommand{\sign}[2][~]{{\small\myrule{#2}\\[-0.7em]\makebox[#2]{\it #1}}}

% Поля
\usepackage[top=20mm, left=30mm, right=10mm, bottom=20mm, nohead]{geometry}
\usepackage{indentfirst}

% Межстрочный интервал
\renewcommand{\baselinestretch}{1.50}

% ------------------------------------------------------------------------------
% minted
% ------------------------------------------------------------------------------
% \usepackage{minted}


% ------------------------------------------------------------------------------
% tcolorbox / tcblisting
% ------------------------------------------------------------------------------
\usepackage{xcolor}
\definecolor{codecolor}{HTML}{FFC300}

\usepackage{tcolorbox}
% \tcbuselibrary{most,listingsutf8,minted}

\tcbset{tcbox width=auto,left=1mm,top=1mm,bottom=1mm,
right=1mm,boxsep=1mm,middle=1pt}

% \newtcblisting{myr}[1]{colback=codecolor!5,colframe=codecolor!80!black,listing only, 
% minted options={numbers=left, style=tcblatex,fontsize=\tiny,breaklines,autogobble,linenos,numbersep=3mm},
% left=5mm,enhanced,
% title=#1, fonttitle=\bfseries,
% listing engine=minted,minted language=r}

%%%%%%%%%%%%%%%%%%%%%%%%%%%%%%%%%%%%%%%

\begin{document}

%%%%%%%%%%%%%%%%%%%%%%%%%%%%%%%
%%%                         %%%
%%% Начало титульного листа %%%

\thispagestyle{empty}
\begin{center}


    \renewcommand{\baselinestretch}{1}
    {\large
        {\sc Петрозаводский государственный университет\\
            Институт математики и информационных технологий\\
            Кафедра информатики и математического обеспечения
        }
    }

\end{center}


\begin{center}
    %%%%%%%%%%%%%%%%%%%%%%%%%
    %
    % Раскомментируйте (уберите знак процента в начале строки)
    % для одной из строк типа направления  - бакалавриат/
    % магистратура и для одной из
    % строк Вашего направление подготовки
    %
    Направление подготовки бакалавриата \\
    % 01.03.02 --- Прикладная математика и информатика \\
    % 09.03.02 --- Информационные системы и технологии \\
    09.03.04 --- Программная инженерия \\
    %%%%%%%%%%%%%%%%%%%%%%%%%
\end{center}

\vfill

\begin{center}

    {\normalsize
        Отчет о проектной работе по курсу <<Разработка приложений для мобильных ОС>>}
    \medskip

    %%% Название работы %%%
    {\Large \sc {Разработка приложения \\ <<Notes>> }} \\
\end{center}

\medskip

\begin{flushright}
    \parbox{11cm}{%
        \renewcommand{\baselinestretch}{1.2}
        \normalsize
        Выполнила:\\
        % Выполнили:\\
        %%% ФИО студента %%%
        студентка 2 курса группы 22207
        \begin{flushright}
            Михеева Валерия Александровна \sign[подпись]{4cm}
        \end{flushright}

        Выполнил:\\
        % Выполнили:\\
        %%% ФИО студента %%%
        студент 2 курса группы 22207
        \begin{flushright}
            Мельников Илья Евгеньевич \sign[подпись]{4cm}
        \end{flushright}
        %%% Второй участник %%%
        % студента 1 курса группы 2210X
        % \begin{flushright}
        % 	И. О. Фамилия \sign[подпись]{4cm}
        % \end{flushright}

        %%%%%%%%%%%%%%%%%%%%%%%%%
        % девушкам применять "Выполнила" и "студентка"
        %%%%%%%%%%%%%%%%%%%%%%%%% 
        
    }
\end{flushright}

\vfill

\begin{center}
    \large
    Петрозаводск --- 2022
\end{center}

%%% Конец титульного листа  %%%
%%%                         %%%
%%%%%%%%%%%%%%%%%%%%%%%%%%%%%%%

%%%%%%%%%%%%%%%%%%%%%%%%%%%%%%%%
%%%                          %%%
%%% Содержание               %%%

\newpage

\tableofcontents

%%% Содержание              %%%
%%%                         %%%
%%%%%%%%%%%%%%%%%%%%%%%%%%%%%%%

%%%%%%%%%%%%%%%%%%%%%%%%%%%%%%%%
%%%                          %%%
%%% Введение                 %%%

%%% В введении Вы должны описать предметную область, с которой связана %%%
%%% Ваша работа, показать её актуальность, вкратце определить цель     %%%
%%% разработки					       %%%


\newpage
\section*{Введение}
\addcontentsline{toc}{section}{Введение}
У электронных блокнотов, на самом деле, никак не посчитать положительных сторон согласно сопоставлению с традиционными бумажными. Они легкодоступны на любых устройствах — от мобильных гаджетов вплоть до личных пк, всегда под рукой и одинаково хорошо подойдут равно как с целью текстовых записей, так и для составления списков дел, записи  безотлагательных идей и мыслей. С их помощью можно удобно организовывать также структурировать информацию, быстро находить нужные заметки, составлять планы, формировать напоминания и объединять все записи в одном приложении.

Цель проекта: разработать приложение <<Notes>> на Java + SQL.

Задачи проекта:
%%% Пример создания списков %%%
\begin{enumerate}
    \item Разработать приложение создания заметок;
    \item Создать удобный пользовательский интерфейс;
    \item Создать базовые функции управления приложением.
\end{enumerate}

%%%                          %%%
%%%%%%%%%%%%%%%%%%%%%%%%%%%%%%%%

%%%%%%%%%%%%%%%%%%%%%%%%%%%%%%%
%%% Требования к приложению %%%
\newpage
\section{Требования к приложению}
 Приложение обладает следующим функционалом: создание/редактирование/удаление дел; отображение дел в виде отсортированного списка; возможность отметить дело, как выполненное.

% \subsection{Подраздел}

%%%                                     %%%
%%%%%%%%%%%%%%%%%%%%%%%%%%%%%%%%%%%%%%%%%%%

%%%%%%%%%%%%%%%%%%%%%%%%%%%%%%%%%%%%%%%%%%%
%%%                                     %%%
%%% Проектирование приложения           %%%
\newpage
\section{Проектирование приложения}


\begin{enumerate}
    \item Для построения приложения используются архитектурные компоненты от Google: Room, ViewModel, LiveData.
   
\end{enumerate}

%%%                          %%%
%%%%%%%%%%%%%%%%%%%%%%%%%%%%%%%%

%%%%%%%%%%%%%%%%%%%%%%%%%%%%%%%%
%%%                          %%%
%%% Реализация приложения    %%%
\newpage
\section{Реализация приложения}

С помощью LiveData создается хранилище данных, работающее по принципу паттерна Observer (наблюдатель). Это хранилище умеет делать две вещи:
1) В него можно поместить какой-либо объект
2) На него можно подписаться и получать объекты, которые в него помещают.
Использование библиотеки Room облегчает работу с объектами SQLiteDatabase в приложении, уменьшая объём стандартного кода и проверяя SQL-запросы во время компиляции.
ViewModel предоставляется доступ к необходимой информации для отображения на экране

%%%                          %%%
%%%%%%%%%%%%%%%%%%%%%%%%%%%%%%%%

%%%%%%%%%%%%%%%%%%%%%%%%%%%%%%%%
%%%                          %%%
%%% Заключение               %%%

\newpage
\section*{Заключение}
\addcontentsline{toc}{section}{Заключение}
Минималистичное решение для истинных ценителей простоты. Notes не рассчитан на сложные записи, дополненные файлами, и позволяет работать лишь с обычными текстовыми заметками, но зато справляется с этим отлично. При разработке приложения учитывались все требования, которыми должна обладать программа, разработанная для использования начинающими пользователями, поэтому интерфейс программы довольно простой и понятный. Получен немалый опыт создания мобильных приложений и работы с Android Studio. 

Финальный вид интерфейса:
\begin{center}
    \includegraphics[scale=0.5]{24cipLpyLqg.jpg}
\end{center}

%%%                          %%%
%%%%%%%%%%%%%%%%%%%%%%%%%%%%%%%%

%%%%%%%%%%%%%%%%%%%%%%%%%%%%%%%%
%%%                          %%%
%%% Приложение               %%%

\newpage
\appendix
%\section*{Приложение}
%\addcontentsline{toc}{section}{Приложение}
%\titleformat{\section}[display]
%  {\normalfont\Large\bfseries}
%  {Приложение\ \thesection}
%  {0pt}{\Large\centering}
%\renewcommand{\thesection}{\Asbuk{section}}

%%%                          %%%
%%%%%%%%%%%%%%%%%%%%%%%%%%%%%%%%
\end{document}